\section{Definiciones}

\textbf{Programa}: Conjunto de objetos que colaboran enviándose mensajes. \\

\textbf{Objeto}: Representación de un ente del dominio del problema. \\

\textbf{Ente}: es todo sobre lo que podamos pensar o hablar. \\

\textbf{Escencia}: es lo que hace que una cosa sea lo que es. \\

\textbf{Claboración}:
\begin{itemize}
\item Sincronicas
\item Dirigidas
\item Siempre existe respuesta
\item El receptor desconoce al emisor
\end{itemize}

\textbf{Contrato}:
\begin{itemize}
	\item Implícito
	\item Explícito
\end{itemize}

\textbf{Excepcion}: Se levanta cuando se rompe el contrato. Cuándo se debe levantar una excepcion depende de que tipo de software se va a diseñar. \\

\textbf{Fail Fast}: Levantar la excepcion lo antes posible para poder reaccionar rapido \\

Maneras de definir un contrato:
\begin{itemize}
\item Precondiciones
\item Postcondiciones
\item Invariantes
\end{itemize}

Metaprogramacion:
\begin{itemize}
\item Estructural - Lectura
\item Estructural - Escritura
\item Comportamiento - Lectura
\item Comportamiento - Escritura
\end{itemize}

\begin{tabular}{|l|c|c|c|}
\hline
					& Especifico de lenguaje	& Concreta o abstracta	& Negocio \\ \hline
Idiom	 			& Si						& Concreta				& Generica \\ \hline
Patrones de diseño 	& No						& Abstractas 			& Generica \\ \hline
Framework 			& Si						& Concretos				& Particular \\ \hline
\end{tabular}

Decorator: Su intencion es agregar funcionalidad ortogonal dinamicamente a un objeto