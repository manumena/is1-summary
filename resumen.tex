% ALGUNOS PAQUETES REQUERIDOS (EN UBUNTU): %
% ========================================
% %
% texlive-latex-base %
% texlive-latex-recommended %
% texlive-fonts-recommended %
% texlive-latex-extra %
% texlive-lang-spanish (en ubuntu 13.10) %
% ******************************************************** %

\documentclass[a4paper]{article}
\usepackage[spanish]{babel}
\usepackage[utf8]{inputenc}
\usepackage{fancyhdr}
\usepackage[pdftex]{graphicx}
\usepackage{sidecap}
\usepackage{caption}
\usepackage{subcaption}
\usepackage{booktabs}
\usepackage{makeidx}
\usepackage{float}
\usepackage{amsmath, amsthm, amssymb}
\usepackage{amsfonts}
\usepackage{sectsty}
\usepackage{wrapfig}
\usepackage{listings}
\usepackage{enumitem}
\usepackage{hyperref}
\usepackage{listings}
\usepackage{listingsutf8}
\usepackage{enumitem}

% Para ver los marcos
% \usepackage{showframe}

\newcommand{\ord}{\ensuremath{\operatorname{O}}}
\newcommand{\nat}{\ensuremath{\mathbb{N}}}
\renewcommand{\thesubsubsection}{\thesubsection.\alph{subsubsection}}

% Lemas, definiciones, etc.
\theoremstyle{remark}
\newtheorem*{obs}{Observación}
\newtheorem*{nota}{Notación}
\newtheorem*{cor}{Corolario}
\theoremstyle{definition}
\newtheorem*{defi}{Definición}
\theoremstyle{plain}
\newtheorem{teo}{Teorema}
\newtheorem{lema}{Lema}
\newtheorem{prop}{Propiedad}

\usepackage{fancyhdr}
% \pagestyle{fancy}
%\renewcommand{\chaptermark}[1]{\markboth{#1}{}}
\renewcommand{\sectionmark}[1]{\markright{\thesection\ - #1}}
\fancyhf{}
% \fancyhead[LO]{Sección \rightmark} % \thesection\

% \fancyfoot[RO]{\thepage}
\renewcommand{\headrulewidth}{0.5pt}
\renewcommand{\footrulewidth}{0.5pt}
%\setlength{\hoffset}{-0.8in}
\setlength{\textwidth}{16cm}
\setlength{\hoffset}{-1.1cm}
\setlength{\headsep}{0.5cm}
\setlength{\textheight}{25cm}
\setlength{\voffset}{-0.7in}
\setlength{\headwidth}{\textwidth}
\setlength{\headheight}{13.1pt}
\renewcommand{\baselinestretch}{1.1} % line spacing


\begin{document}

\title{Ingeniería de software I}
\author{Manuel Mena}
\maketitle

\tableofcontents

\newpage
\section{Definiciones}

\textbf{Programa}: Conjunto de objetos que colaboran enviándose mensajes. \\

\textbf{Objeto}: Representación de un ente del dominio del problema. \\

\textbf{Ente}: es todo sobre lo que podamos pensar o hablar. \\

\textbf{Escencia}: es lo que hace que una cosa sea lo que es. \\

\textbf{Claboración}:
\begin{itemize}
\item Sincronicas
\item Dirigidas
\item Siempre existe respuesta
\item El receptor desconoce al emisor
\end{itemize}

\textbf{Contrato}:
\begin{itemize}
	\item Implícito
	\item Explícito
\end{itemize}

\textbf{Excepcion}: Se levanta cuando se rompe el contrato. Cuándo se debe levantar una excepcion depende de que tipo de software se va a diseñar. \\

\textbf{Fail Fast}: Levantar la excepcion lo antes posible para poder reaccionar rapido \\

Maneras de definir un contrato:
\begin{itemize}
\item Precondiciones
\item Postcondiciones
\item Invariantes
\end{itemize}

Metaprogramacion:
\begin{itemize}
\item Estructural - Lectura
\item Estructural - Escritura
\item Comportamiento - Lectura
\item Comportamiento - Escritura
\end{itemize}

\begin{tabular}{|l|c|c|c|}
\hline
					& Especifico de lenguaje	& Concreta o abstracta	& Negocio \\ \hline
Idiom	 			& Si						& Concreta				& Generica \\ \hline
Patrones de diseño 	& No						& Abstractas 			& Generica \\ \hline
Framework 			& Si						& Concretos				& Particular \\ \hline
\end{tabular}

Decorator: Su intencion es agregar funcionalidad ortogonal dinamicamente a un objeto

\end{document}
